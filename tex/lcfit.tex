\documentclass{amsart}

\usepackage[round]{natbib}
\usepackage{graphicx}
\usepackage[notref,notcite]{showkeys}
\usepackage{url}
\usepackage{xr}

\newcommand{\llf}{\mathcal{L}}   % the log likelihood function
\newcommand{\llfs}{\tilde{\llf}}
\newcommand{\loglike}{\ell}           % an observed log likelihood

\begin{document}

\section{A series expansion for the simplest phylogenetic likelihood function}

Here is our log likelihood function:
\begin{equation}
  \llf(c,m,r,t) = c \log\left(\frac{1}{2}(1+e^{-rt})\right) + m \log\left(\frac{1}{2}(1-e^{-rt})\right).
  \label{eq:llf}
\end{equation}
Below we will be fitting \eqref{eq:llf} to a series of log likelihood observations.
It's clear that the absolute scale of mutation does not matter.
That is, $\llf(c,m,r,t) = \lambda^{-1} \llf(\lambda c,\lambda m,r,t)$

With that in mind, our first step is to divide through by $m$, and define $\rho = c/m$ so that
\begin{equation}
  \llfs(\rho,r,t) := \llf(\rho,1,r,t) = \rho \log \left(\frac{1}{2}(1+e^{-rt})\right) + \log \left(\frac{1}{2}(1-e^{-rt})\right).
  \label{eq:llfs}
\end{equation}

We will develop a series expansion for \eqref{eq:llfs} by considering each term individually.
One can use a standard Taylor series for the first term:
\begin{equation}
  \log \left(\frac{1}{2}(1+e^{-rt})\right) = - \frac{rt}{2} + \frac{r^2 t^2}{8} + O(x^4).
\end{equation}
The second term is first broken down into two components:
\begin{equation}
  \log \left(\frac{1}{2}(1-e^{-rt})\right) = \log \left(\frac{rt}{2} \right) + \log \left(\frac{1-e^{-rt}}{2 rt}\right).
\end{equation}

This second term can be expanded according to Taylor's theorem, most easily by expanding the expansion
\[
  \log \left(\frac{1-e^{-rt}}{2 rt}\right) = \log \left(1 - \frac{rt}{2!} + \frac{r^2 t^2}{3!} - \frac{r^3 t^3}{4!} + O(t^4) \right)
\]
to get
\[
  \log \left(\frac{1-e^{-rt}}{2 rt}\right) = - \frac{rt}{2} + \frac{r^2 t^2}{24} + O(t^4).
\]

This means that \eqref{eq:llfs} becomes
\[
  \begin{split}
    \llfs(\rho,r,t) & = \rho \cdot \left(-\frac{rt}{2} + \frac{r^2 t^2}{8} + O(x^4) \right) + \log \left(\frac{rt}{2} \right) - \frac{rt}{2} + \frac{r^2 t^2}{24} + O(t^4)\\
    & = \log(r/2) + \log(t) - \frac{1+\rho}{2} r t + \frac{1+3 \rho}{24} r^2 t^2 + O(t^4).
  \end{split}
\]

Define
\begin{equation}
  \llfs_3(\rho,r,t) := \log(r/2) + \log(t) - \frac{1+\rho}{2} r t + \frac{1+3 \rho}{24} r^2 t^2.
  \label{eq:llfs3}
\end{equation}

\section{Fitting $\llfs_3$}

Assume we are given three distinct $(t_i, \loglike_i)$ pairs, where $t_i$ is a branch length, and $\loglike_i$ is the corresponding phylogenetic log likelihood, we would like to find $c$, $m$, and $r$ such that \eqref{eq:llf} goes through those three points.
Given a pair (branch lengths, likelihood value) pairs, we would like to fit \eqref{eq:llfs3}.


Taking the difference between two of these points gives.
\begin{equation}
  \loglike_1 - \loglike_2 = \frac{(1+3 \rho)(t_1^2 - t_2^2)}{24} r^2 - \frac{(1+\rho) (t_1 - t_2)}{2} r + \log t_1 - \log t_2.
  \label{eq:forc}
\end{equation}


An application of the quadratic formula leads to these solutions for \eqref{eq:forc}:
\begin{equation}
  \begin{split}
  r & = \frac{\frac{(1+\rho) (t_1 - t_2)}{2} \pm \sqrt{\frac{(1+\rho)^2 (t_1 - t_2)^2}{4} - 4 \frac{(1+3 \rho)(t_1^2 - t_2^2)}{24} (\log t_1 - \log t_2)}}
  {2 \frac{(1+3 \rho)(t_1^2 - t_2^2)}{24}} \\
    & = 6 \ \frac{(1+\rho) (t_1 - t_2) \pm \sqrt{(1+\rho)^2 (t_1 - t_2)^2 - \frac{2}{3} (1+3 \rho)(t_1^2 - t_2^2) (\log t_1 - \log t_2)}}
	  {(1+3 \rho)(t_1^2 - t_2^2)} \\
    & = 6 \ \frac{(1+\rho) \pm \sqrt{(1+\rho)^2 - \frac{2}{3} (1+3 \rho)(t_1 + t_2) (\log t_1 - \log t_2) / (t_1 - t_2)}}
	  {(1+3 \rho)(t_1 + t_2)}.
  \end{split}
  \label{eq:forc}
\end{equation}
Using the $(t_1, \loglike_1)$ and $(t_2, \loglike_2)$ we can think of the larger solution of this equation $r$ as a function of $\rho$.

Then we can put this $r(\rho)$ back into \eqref{eq:llfs3} and solve for a $\rho$ using $(t_3, \loglike_3)$:
\begin{equation}
  \loglike_3 = \log(r(\rho)/2) + \log(t_3) - \frac{1+\rho}{2} r(\rho) t_3 + \frac{1+3 \rho}{24} \left[r(\rho) t_3 \right]^2.
\end{equation}

In order for this to evaluate to a real value, we need for $r(\rho)$ to evaluate to a positive real value.
That translates to
\begin{gather}
  0 < (1+\rho)^2 - \frac{2}{3} (1+3 \rho)(t_1 + t_2) (\log t_1 - \log t_2) / (t_1 - t_2) \label{eq:torootpos} \\
  0 < (1+\rho) + \sqrt{(1+\rho)^2 - \frac{2}{3} (1+3 \rho)(t_1 + t_2) (\log t_1 - \log t_2) / (t_1 - t_2)} \label{eq:rhopos}
\end{gather}

Note that \eqref{eq:rhopos} is always satisfied if \eqref{eq:torootpos} is.
Now, \eqref{eq:torootpos} is equivalent to
\begin{gather}
  \frac{2}{3} (1+3 \rho)(t_1 + t_2) (\log t_1 - \log t_2) / (t_1 - t_2) < (1+\rho)^2 \\
  \frac{2}{3} (1+3 \rho)(t_1 + t_2) (\log t_1 - \log t_2) < (t_1 - t_2) (1+\rho)^2 \\
  \frac{2 (1+3 \rho)}{3 (1+\rho)^2} < \frac{t_1 - t_2}{(t_1 + t_2) (\log t_1 - \log t_2)}
\end{gather}

Note that the RHS is positive because $t_1 > t_2$ is equivalent to $\log t_1 > \log t_2$.

\end{document}


\noindent\(\pmb{\text{forc} =\text{FullSimplify}[s\text{/.}\text{rsol}]}\)

\noindent\(\log(t)-\frac{(1+c) t \left(3 (1+c) t_1-3 (1+c) t_2+\sqrt{3} \sqrt{(t_1-t_2) \left(3 (1+c)^2 (t_1-t_2)+2
(1+3 c) (t_1+t_2) (\loglike_1-\loglike_2-\log(t_1)+\log(t_2))\right)}\right)}{(1+3 c) (t_1-t_2) (t_1+t_2)}+\frac{\left(3
(1+c) t (t_1-t_2)+\sqrt{3} t \sqrt{(t_1-t_2) \left(3 (1+c)^2 (t_1-t_2)+2 (1+3 c) (t_1+t_2) (\loglike_1-\loglike_2-\log(t_1)+\log(t_2))\right)}\right)^2}{6
(1+3 c) \left(t_1^2-t_2^2\right)^2}+\text{Log}\left[\frac{3 (1+c) t_1-3 (1+c) t_2+\sqrt{3} \sqrt{(t_1-t_2) \left(3
(1+c)^2 (t_1-t_2)+2 (1+3 c) (t_1+t_2) (\loglike_1-\loglike_2-\log(t_1)+\log(t_2))\right)}}{(1+3 c)
yt_1-t_2) (t_1+t_2)}\right]\)



